\documentclass[a4paper]{easychair}%[11pt]{llncs}
\usepackage[utf8]{inputenc}
\title{Invited Talk: ML applications to string theory}
\author{Fabian Ruhle}
\institute{CERN}

\begin{document}
\maketitle
\begin{abstract}
String theory has evolved into a complex framework used to address and advance open problems in physics and mathematics. Recently, machine learning techniques have been applied to address problems arising in this context. In this talk, I will provide an overview of these recent developments. In more detail, many properties of string theory are topological in nature and can hence be described by discrete mathematical data. This often results in computationally hard combinatorial problems or in Diophantine equations, which have been successfully addressed using Reinforcement Learning, genetic algorithms, and SAT solvers.
Other properties lead to differential equations that depend on continuous properties of space-time. These have been addressed using fast optimizers and differentiation provided by ML libraries.


\end{abstract}

\end{document}
